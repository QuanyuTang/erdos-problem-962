\documentclass[11pt,letterpaper,reqno]{amsart}
\usepackage{tikz}
\usetikzlibrary{positioning, shapes.geometric, arrows.meta, calc, positioning}
\usepackage{amssymb}
\usepackage{amsmath}
\usepackage{amsthm}
\usepackage{amsfonts}
\usepackage{bbm}
\usepackage{enumitem} 
\usepackage{pgfplots}
\pgfplotsset{compat=1.18} 
\usepackage{booktabs}
\usepackage{graphicx}
\usepackage[T1]{fontenc}
\usepackage{doi}
\usepackage{float} 
\addtolength{\hoffset}{-1.5cm}\addtolength{\textwidth}{3cm}
\addtolength{\voffset}{-1cm}\addtolength{\textheight}{2cm}

\usepackage{bookmark}
\usepackage{hyperref}
\hypersetup{pdfstartview={FitH}}
\newcommand{\C}{\mathbb{C}}
\newcommand{\cE}{\mathcal{E}}
\newcommand{\norm}[1]{\lVert #1 \rVert}
\newcommand{\abs}[1]{| #1 |}
\newcommand{\bv}{\mathbf{v}}
\newcommand{\bw}{\mathbf{w}}
\newcommand{\tr}{\operatorname{Tr}}
\DeclareMathOperator{\rank}{rank}




\newtheorem{theorem}{Theorem}[section]
\newtheorem{lemma}[theorem]{Lemma}
\newtheorem{proposition}[theorem]{Proposition}
\newtheorem{corollary}[theorem]{Corollary}
\newtheorem{claim}{Claim}
\newtheorem{question}[theorem]{Question}
\newtheorem{problem}[theorem]{Problem}
\newtheorem{conjecture}[theorem]{Conjecture}
\theoremstyle{definition}
\newtheorem{example}[theorem]{Example}
\newtheorem{remark}[theorem]{Remark}
\newtheorem{definition}[theorem]{Definition}
\numberwithin{equation}{section}
\newcommand{\NN}{\mathbb{N}}
\newcommand{\taufunc}{\tau}
\newcommand{\omegap}{\omega}
\newcommand{\ord}{\operatorname{ord}}
\newcommand{\R}{\mathbb{R}}        % real numbers
\newcommand{\E}{\mathbb{E}}        % expectation
\newcommand{\Var}{\mathrm{Var}}    % variance
\newcommand{\Cov}{\operatorname{Cov}}
\newcommand{\PP}{\mathbb{P}}     % probability
\newcommand{\eps}{\varepsilon}     % epsilon
\newcommand{\ind}{\mathbf{1}}      % indicator function
\newcommand{\seq}[1]{\left(#1\right)} % sequence
\newcommand{\lcm}{\operatorname{lcm}}
\newcommand{\Pmax}{P^{+}} % largest prime factor
\newcommand{\PsiSmooth}{\Psi} % smooth number counting function
\makeatother


\begin{document}

\title{An improved lower bound on Erd\H{o}s Problem \#962}



\author[Quanyu Tang]{Quanyu Tang}
\address{School of Mathematics and Statistics, Xi'an Jiaotong University, Xi'an 710049, P. R. China}
\email{tang\_quanyu@163.com}


% \subjclass[2020]{xxx}

% \keywords{xxx}

% \begin{abstract}

% \end{abstract}



\maketitle









\section{Introduction}

In \cite[p.~96]{Er65}, Erd\H{o}s proposed the following problem, which also appears as Problem~\#962 on Bloom’s Erd\H{o}s Problems website~\cite{EP962}.

\begin{problem}
Let $k(n)$ be the maximal integer $k$ such that there exists $m\leq n$ such that each of the integers\[m+1,\ldots,m+k\]are divisible by at least one prime $>k$. Estimate $k(n)$.
\end{problem}

Erd\H{o}s~\cite{Er65} wrote it is 'not hard to prove' that\[k(n)\gg_\epsilon \exp\left((\log n)^{1/2-\epsilon}\right)\]and it 'seems likely' that $k(n)=o(n^\epsilon)$, but had no non-trivial upper bound for $k(n)$.

It is not clear what he meant by a non-trivial bound for this problem, but Tao in the comments section of~\cite{EP962} has given a simple argument proving $k(n) \leq (1+o(1))n^{1/2}$.


In this short note we prove a new lower bound for $k(n)$.

\begin{theorem}
As $n\to\infty$, we have
\[
k(n) \ge \exp \Big(\big(\tfrac1{\sqrt2}-o(1)\big)\sqrt{\log n\,\log\log n}\Big).
\]
\end{theorem}


\section{Main result}

\begin{theorem}\label{thm:kn-lower}
For every fixed constant $c$ with $0<c<1/\sqrt{2}$ there exists
$n_0(c)$ such that for all integers $n\ge n_0(c)$,
\[
k(n)\ \ge\ \left\lfloor \exp\!\big(c\sqrt{\log n\,\log\log n}\big)\right\rfloor.
\]
\end{theorem}

\begin{proof}
For an integer $t\ge 1$, let $\Pmax(t)$ denote the largest prime divisor of $t$
(with the convention $\Pmax(1)=1$). For real numbers $x\ge 1$ and $y\ge 2$, define
\[
\PsiSmooth(x,y)\ :=\ \#\{\,t\le x:\ \Pmax(t)\le y\,\},
\]
the counting function of $y$--smooth integers up to $x$.

\medskip\noindent
\textbf{1. A pigeonhole lemma:}
Fix an integer $y\ge 2$ and set $Q:=\lfloor n/y\rfloor$. Partition the interval
$\{1,2,\dots,Qy\}$ into $Q$ disjoint blocks of length $y$:
\[
B_j:=\{jy+1,\dots,(j+1)y\}\qquad (j=0,1,\dots,Q-1).
\]
If every block $B_j$ contains at least one $y$--smooth integer, then
$\PsiSmooth(n,y)\ge Q$. Hence, if $\PsiSmooth(n,y)<Q$, there exists some $j$
such that $B_j$ contains no $y$--smooth integer, i.e.\ $\Pmax(t)>y$ for all
$t\in B_j$. Taking $m:=jy$, we have $m\le Qy\le n$ and
\[
\Pmax(m+i)>y \qquad (1\le i\le y).
\]
In particular, each $m+i$ is divisible by a prime $>y$.

Thus, to prove the theorem it suffices to find $y$ with $\PsiSmooth(n,y)<\lfloor n/y\rfloor$.

\medskip\noindent
\textbf{2. Choosing $y$ and estimating $\PsiSmooth(n,y)$:}
Fix $c\in(0,1/\sqrt2)$ and set
\[
y:=\left\lfloor \exp\!\big(c\sqrt{\log n\,\log\log n}\big)\right\rfloor,
\qquad
u:=\frac{\log n}{\log y}.
\]
Then $y\to\infty$ and $u\to\infty$ as $n\to\infty$, and moreover
$\log y \sim c\sqrt{\log n\,\log\log n}$, hence
\[
u=\frac{\log n}{\log y}
=\frac{1}{c}\sqrt{\frac{\log n}{\log\log n}}\,(1+o(1)),
\qquad
\log u=\frac12\log\log n+O(\log\log\log n),
\]
so
\begin{equation}\label{eq:uloguasymp}
u\log u
=\Big(\tfrac{1}{2c}+o(1)\Big)\sqrt{\log n\,\log\log n}.
\end{equation}

A classical estimate of de Bruijn, with Hildebrand's extension of the uniformity
range, gives
\begin{equation}\label{eq:Psi-rho}
\PsiSmooth(n,y)=n\,\rho(u)\Big(1+O \Big(\frac{\log(u+1)}{\log y}\Big)\Big),
\end{equation}holds for $y > \exp \left( (\log\log n)^{5/3 + \varepsilon} \right)$, where $\rho(u)$ is the Dickman--de Bruijn function; see \cite[(1.8)--(1.10)]{Granville2008}.
In our choice of parameters, $\log(u+1)/\log y\to 0$, so
\[
\PsiSmooth(n,y)=n\,\rho(u)\,(1+o(1)).
\]
Moreover, $\rho(u)$ decays as
\begin{equation}\label{eq:rho-asymp}
\rho(u)=u^{-u+o(u)}=\exp\big(-(1+o(1))u\log u\big)
\qquad (u\to\infty),
\end{equation}
see \cite[(1.6)]{Granville2008}.
Combining \eqref{eq:uloguasymp} and \eqref{eq:rho-asymp} yields
\[
\PsiSmooth(n,y)
\le n\exp\!\Big(-\Big(\tfrac{1}{2c}+o(1)\Big)\sqrt{\log n\,\log\log n}\Big).
\]

\medskip\noindent
\textbf{3. Comparison with $n/y$:}
Since $\log y = c\sqrt{\log n\,\log\log n}+O(1)$, we have
\[
\frac{n}{y}
= n\exp\!\big(-c\sqrt{\log n\,\log\log n}+O(1)\big).
\]
Because $c<1/\sqrt2$, we have $\frac{1}{2c}-c>0$; choose a constant
$\delta>0$ with $\delta<\frac{1}{2c}-c$. Then for all sufficiently large $n$,
the previous bounds imply
\[
\PsiSmooth(n,y)\ \le\ \frac{n}{y}\,\exp\!\big(-\delta\sqrt{\log n\,\log\log n}\big)
\ <\ \frac{1}{2}\cdot\frac{n}{y}
\ \le\ \left\lfloor\frac{n}{y}\right\rfloor.
\]
Therefore $\PsiSmooth(n,y)<\lfloor n/y\rfloor$, and by Step~1 there exists
$m\le n$ such that each of $m+1,\dots,m+y$ has largest prime factor $>y$, hence
is divisible by a prime $>y$. This shows $k(n)\ge y$.
\end{proof}



\begin{thebibliography}{99}


\bibitem{EP962}
T. F. Bloom, Erd\H{o}s Problem \#962, \url{https://www.erdosproblems.com/962}, accessed 2025-12-28.



\bibitem{Er65}
Erd\H{o}s, P., Extremal problems in number theory. \emph{Proc. Sympos. Pure Math.}, Vol. VIII (1965), 181--189.




\bibitem{Granville2008}
A.~Granville, \emph{Smooth numbers: computational number theory and beyond}, in \emph{Algorithmic Number Theory: Lattices, Number Fields, Curves and Cryptography}
(MSRI Publications, Vol.~44), J.~P. Buhler and P.~Stevenhagen (eds.), Cambridge University Press, 2008, pp.~267--324. \texttt{doi:10.1017/9781139049801.010}.

\end{thebibliography}













\end{document}

